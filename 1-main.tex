\documentclass{article}
\usepackage{geometry}
\usepackage{titling}
\usepackage{fancyhdr}
\pagenumbering{arabic}

\setcounter{page}{1}
\fontsize{25}{13.5}
\date{}
 \title{
\textbf{Gravedad en las escalas más pequeñas: La Longitud Mínima como base para construir una Teoría de Gravedad Cuántica.}
}
\author{Por: 
Maria Fernanda Martínez Vázquez}
 \geometry{
 letterpaper,
 total={170mm,257mm},
 bottom = 30mm,
 footskip = 10mm,
 top=25mm,
 headsep=-7mm
 }

\begin{document}
\vspace{-3pt}
\maketitle
\begin{center}
\vspace{-48pt}
     Profesor: Dr. Luis Antonio Arias Hernández\\
     Materia: Física Estadística   

\end{center}


\fancypagestyle{plain}{%  the preset of fancyhdr 
    \fancyhf{} % clear all header and footer fields
    
    \renewcommand{\footrulewidth}{0.4pt}
    \fancyfoot[L]{Octubre 5, 2024}
    \fancyhead[L]{Longitud Mínima}
    \fancyhead[R]{\theauthor}
    \fancyfoot[R]{\thepage}
}

\section*{Introducción}

El artículo trata sobre la idea de una \textbf{longitud mínima} como una posible clave para unificar la mecánica cuántica y la relatividad general, un desafío en la formulación de la gravedad cuántica. La longitud mínima sugiere que el espacio-tiempo no puede dividirse indefinidamente, resolviendo paradojas como la formación de agujeros negros en alta energía. Este concepto transforma la idea del espacio-tiempo continuo, haciendo que adquiera una estructura granular, lo cual podría eliminar singularidades como las de los agujeros negros o el Big Bang.

La longitud mínima tiene efectos detectables en la teoría cuántica, como correcciones en problemas tradicionales, lo que abre vías experimentales para probar su existencia. Aunque más accesible que teorías como la de cuerdas, enfrenta el reto de ser difícil de verificar experimentalmente debido a las escalas involucradas. Sin embargo, avances futuros podrían ayudar a detectarla, por ejemplo, en fenómenos astrofísicos como la radiación de los agujeros negros.

\section*{El Principio de Incertidumbre Generalizado}

El **principio de incertidumbre generalizado** (GUP, por sus siglas en inglés) extiende el principio de incertidumbre de Heisenberg en el contexto de teorías cuánticas de la gravedad. En su forma tradicional, el principio de incertidumbre de Heisenberg establece que no es posible conocer simultáneamente la posición y el momento de una partícula con precisión arbitraria. Sin embargo, en el GUP, esta relación se modifica para incluir un término adicional proporcional al cuadrado de la incertidumbre en el momento, lo que introduce un límite fundamental en la precisión con la que se puede medir la posición de una partícula. Este límite implica la existencia de una \textbf{longitud mínima} en la naturaleza, por lo que hay una resolución máxima en las mediciones espaciales, lo que sugiere que el espacio-tiempo podría estar discretizado a escalas extremadamente pequeñas.

Este principio surge en varios enfoques teóricos como la teoría de cuerdas y la gravedad cuántica, y apunta a que, en presencia de fuertes efectos gravitacionales, la energía necesaria para sondear distancias muy pequeñas podría generar un agujero negro, impidiendo cualquier medición más allá de cierto límite.



\section*{Geometría no conmutativa}

La \textbf{geometría no conmutativa} (NCG, por sus siglas en inglés) es una extensión de la geometría clásica en la cual las coordenadas espaciales ya no conmutan entre sí. Esto se formaliza mediante la modificación del conmutador entre los operadores de posición, de modo que:

\[
[x̂_\mu, x̂_\nu] = iθ_{\mu\nu}
\]

donde \( θ_{\mu\nu} \) es una matriz antisimétrica cuyas entradas tienen dimensiones de área, y controlan el comportamiento no conmutativo de las coordenadas espaciales.

En la geometría clásica, las posiciones de puntos en el espacio conmutan, lo que significa que medir la coordenada \( x \) antes o después de la coordenada \( y \) da el mismo resultado. Sin embargo, en la geometría no conmutativa, este orden de medición afecta el resultado debido a que las posiciones ya no conmutan, lo que refleja un comportamiento cuántico del espacio-tiempo a escalas extremadamente pequeñas.

Una implementación común de la NCG se basa en el \textbf{producto de Moyal}, que reemplaza el producto usual de funciones en el espacio-tiempo por un producto no conmutativo, descrito matemáticamente como:

$$
(f \star g)(x) = \int d^4y d^4k \frac{f(x_\mu + \frac{1}{2}\theta_{\mu\nu}k_\nu)g(x_\mu + y_\mu)e^{ik_\nu y^\nu}}{(2\pi)^4}
$$

El enfoque de la NCG propone que el espacio-tiempo en sí mismo adquiere una estructura no conmutativa a pequeñas escalas, lo cual podría ser relevante para el entendimiento de la gravedad cuántica y fenómenos que involucran energías extremadamente altas, como los agujeros negros o la física en colisionadores.

\section*{Resultados}

El artículo presenta los resultados obtenidos al aplicar dos enfoques principales para modelar la existencia de una \textbf{longitud mínima} en el contexto de la gravedad cuántica: el \textbf{Principio de Incertidumbre Generalizado} (GUP) y la \textbf{Geometría No Conmutativa} (NCG). A continuación, se resumen los principales hallazgos:

\textbf{1. Principio de Incertidumbre Generalizado (GUP):}
   - El GUP introduce una modificación en la relación de incertidumbre de Heisenberg, que lleva a la aparición de una longitud mínima. Esto implica que no se puede medir la posición con precisión arbitraria, ya que hay un límite fundamental en la resolución espacial.
   - Aplicaron el GUP a problemas clásicos de mecánica cuántica, como el oscilador armónico y el átomo de hidrógeno, y encontraron que la longitud mínima induce pequeñas modificaciones en los resultados estándar.
   - En procesos de dispersión, como la aniquilación \(e^+e^-\), la sección eficaz del Modelo Estándar se reduce debido a la inclusión de la longitud mínima.

\textbf{2. Geometría No Conmutativa (NCG):}
   - La NCG se enfoca en modificar el conmutador entre las posiciones espaciales, lo que genera un espacio-tiempo con una estructura no conmutativa. Esto también introduce una longitud mínima y modifica las soluciones clásicas de problemas en mecánica cuántica.
   - En el contexto de los agujeros negros, encontraron que las soluciones de agujeros negros no conmutativos evitan las singularidades clásicas, ofreciendo soluciones que son regulares en todo el espacio y tienen una temperatura finita a lo largo de su evaporación. Esto elimina las divergencias presentes en la teoría semiclásica.

\textbf{3. Aplicaciones Fenomenológicas:}
   - Aunque los efectos de la longitud mínima son pequeños, estudios realizados en colisionadores como el LHC podrían, en el futuro, imponer límites experimentales sobre la longitud mínima.
   - Aplicaron el modelo GUP a la oscilación de neutrinos, observando pequeñas modificaciones en los patrones de oscilación, pero concluyeron que los efectos son demasiado pequeños para ser detectados con los experimentos actuales. Sin embargo, fenómenos astrofísicos como los \textbf{estallidos de rayos gamma} podrían proporcionar una vía para observar estos efectos a energías ultraaltas.



\section*{Conclusiones}

En las conclusiones del artículo, los autores destacan varios puntos clave sobre sus hallazgos y las implicaciones de sus estudios:

 \textbf{1. Modelos efectivos de gravedad cuántica:} Los enfoques del \textbf{Principio de Incertidumbre Generalizado (GUP)} y la \textbf{Geometría No Conmutativa (NCG)} son modelos efectivos que permiten estudiar la aparición de una \textbf{longitud mínima} en el contexto de la gravedad cuántica. Aunque estas teorías son simplificaciones de posibles teorías completas de la gravedad cuántica, ofrecen la ventaja de hacer predicciones independientes de cualquier teoría fundamental específica.

\textbf{2. Predicciones fenomenológicas:} A pesar de que los efectos de una longitud mínima son generalmente muy pequeños, los autores subrayan que estos modelos pueden llevar a predicciones observables en experimentos de alta energía, como los llevados a cabo en el \textbf{LHC} y otros experimentos futuros. De este modo, se podrían poner límites experimentales a la longitud mínima en el espacio-tiempo.

3. \textbf{Simplificación y accesibilidad:} Los modelos GUP y NCG son accesibles y menos complicados que teorías más completas como la \textbf{teoría de cuerdas} o la \textbf{gravedad cuántica de lazos}, lo que los hace ideales para ser utilizados en estudios fenomenológicos sin necesidad de aplicar formalismos matemáticos complejos.

4. \textbf{Potencial futuro}: Aunque los efectos de la longitud mínima son pequeños y difíciles de detectar con la tecnología actual, los autores creen que experimentos futuros, tanto en aceleradores de partículas como en observaciones astrofísicas, podrían proporcionar datos que pongan a prueba las predicciones de estos modelos. Fenómenos astrofísicos como los \textbf{estallidos de rayos gamma} o los \textbf{núcleos galácticos activos} podrían ser fuentes de partículas ultraenergéticas capaces de revelar estos efectos.

En resumen, los autores concluyen que los enfoques efectivos que han desarrollado permiten estudiar la \textbf{longitud mínima} con predicciones observables, y que experimentos futuros serán clave para explorar y verificar estos efectos en la naturaleza.


\end{document}