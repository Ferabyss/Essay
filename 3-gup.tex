El **principio de incertidumbre generalizado** (GUP, por sus siglas en inglés) extiende el principio de incertidumbre de Heisenberg en el contexto de teorías cuánticas de la gravedad. En su forma tradicional, el principio de incertidumbre de Heisenberg establece que no es posible conocer simultáneamente la posición y el momento de una partícula con precisión arbitraria. Sin embargo, en el GUP, esta relación se modifica para incluir un término adicional proporcional al cuadrado de la incertidumbre en el momento, lo que introduce un límite fundamental en la precisión con la que se puede medir la posición de una partícula. Este límite implica la existencia de una \textbf{longitud mínima} en la naturaleza, por lo que hay una resolución máxima en las mediciones espaciales, lo que sugiere que el espacio-tiempo podría estar discretizado a escalas extremadamente pequeñas.

Este principio surge en varios enfoques teóricos como la teoría de cuerdas y la gravedad cuántica, y apunta a que, en presencia de fuertes efectos gravitacionales, la energía necesaria para sondear distancias muy pequeñas podría generar un agujero negro, impidiendo cualquier medición más allá de cierto límite.

