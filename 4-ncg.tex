La \textbf{geometría no conmutativa} (NCG, por sus siglas en inglés) es una extensión de la geometría clásica en la cual las coordenadas espaciales ya no conmutan entre sí. Esto se formaliza mediante la modificación del conmutador entre los operadores de posición, de modo que:

\[
[x̂_\mu, x̂_\nu] = iθ_{\mu\nu}
\]

donde \( θ_{\mu\nu} \) es una matriz antisimétrica cuyas entradas tienen dimensiones de área, y controlan el comportamiento no conmutativo de las coordenadas espaciales.

En la geometría clásica, las posiciones de puntos en el espacio conmutan, lo que significa que medir la coordenada \( x \) antes o después de la coordenada \( y \) da el mismo resultado. Sin embargo, en la geometría no conmutativa, este orden de medición afecta el resultado debido a que las posiciones ya no conmutan, lo que refleja un comportamiento cuántico del espacio-tiempo a escalas extremadamente pequeñas.

Una implementación común de la NCG se basa en el \textbf{producto de Moyal}, que reemplaza el producto usual de funciones en el espacio-tiempo por un producto no conmutativo, descrito matemáticamente como:

$$
(f \star g)(x) = \int d^4y d^4k \frac{f(x_\mu + \frac{1}{2}\theta_{\mu\nu}k_\nu)g(x_\mu + y_\mu)e^{ik_\nu y^\nu}}{(2\pi)^4}
$$

El enfoque de la NCG propone que el espacio-tiempo en sí mismo adquiere una estructura no conmutativa a pequeñas escalas, lo cual podría ser relevante para el entendimiento de la gravedad cuántica y fenómenos que involucran energías extremadamente altas, como los agujeros negros o la física en colisionadores.