El artículo presenta los resultados obtenidos al aplicar dos enfoques principales para modelar la existencia de una \textbf{longitud mínima} en el contexto de la gravedad cuántica: el \textbf{Principio de Incertidumbre Generalizado} (GUP) y la \textbf{Geometría No Conmutativa} (NCG). A continuación, se resumen los principales hallazgos:

\textbf{1. Principio de Incertidumbre Generalizado (GUP):}
   - El GUP introduce una modificación en la relación de incertidumbre de Heisenberg, que lleva a la aparición de una longitud mínima. Esto implica que no se puede medir la posición con precisión arbitraria, ya que hay un límite fundamental en la resolución espacial.
   - Aplicaron el GUP a problemas clásicos de mecánica cuántica, como el oscilador armónico y el átomo de hidrógeno, y encontraron que la longitud mínima induce pequeñas modificaciones en los resultados estándar.
   - En procesos de dispersión, como la aniquilación \(e^+e^-\), la sección eficaz del Modelo Estándar se reduce debido a la inclusión de la longitud mínima.

\textbf{2. Geometría No Conmutativa (NCG):}
   - La NCG se enfoca en modificar el conmutador entre las posiciones espaciales, lo que genera un espacio-tiempo con una estructura no conmutativa. Esto también introduce una longitud mínima y modifica las soluciones clásicas de problemas en mecánica cuántica.
   - En el contexto de los agujeros negros, encontraron que las soluciones de agujeros negros no conmutativos evitan las singularidades clásicas, ofreciendo soluciones que son regulares en todo el espacio y tienen una temperatura finita a lo largo de su evaporación. Esto elimina las divergencias presentes en la teoría semiclásica.

\textbf{3. Aplicaciones Fenomenológicas:}
   - Aunque los efectos de la longitud mínima son pequeños, estudios realizados en colisionadores como el LHC podrían, en el futuro, imponer límites experimentales sobre la longitud mínima.
   - Aplicaron el modelo GUP a la oscilación de neutrinos, observando pequeñas modificaciones en los patrones de oscilación, pero concluyeron que los efectos son demasiado pequeños para ser detectados con los experimentos actuales. Sin embargo, fenómenos astrofísicos como los \textbf{estallidos de rayos gamma} podrían proporcionar una vía para observar estos efectos a energías ultraaltas.

