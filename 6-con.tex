En las conclusiones del artículo, los autores destacan varios puntos clave sobre sus hallazgos y las implicaciones de sus estudios:

 \textbf{1. Modelos efectivos de gravedad cuántica:} Los enfoques del \textbf{Principio de Incertidumbre Generalizado (GUP)} y la \textbf{Geometría No Conmutativa (NCG)} son modelos efectivos que permiten estudiar la aparición de una \textbf{longitud mínima} en el contexto de la gravedad cuántica. Aunque estas teorías son simplificaciones de posibles teorías completas de la gravedad cuántica, ofrecen la ventaja de hacer predicciones independientes de cualquier teoría fundamental específica.

\textbf{2. Predicciones fenomenológicas:} A pesar de que los efectos de una longitud mínima son generalmente muy pequeños, los autores subrayan que estos modelos pueden llevar a predicciones observables en experimentos de alta energía, como los llevados a cabo en el \textbf{LHC} y otros experimentos futuros. De este modo, se podrían poner límites experimentales a la longitud mínima en el espacio-tiempo.

3. \textbf{Simplificación y accesibilidad:} Los modelos GUP y NCG son accesibles y menos complicados que teorías más completas como la \textbf{teoría de cuerdas} o la \textbf{gravedad cuántica de lazos}, lo que los hace ideales para ser utilizados en estudios fenomenológicos sin necesidad de aplicar formalismos matemáticos complejos.

4. \textbf{Potencial futuro}: Aunque los efectos de la longitud mínima son pequeños y difíciles de detectar con la tecnología actual, los autores creen que experimentos futuros, tanto en aceleradores de partículas como en observaciones astrofísicas, podrían proporcionar datos que pongan a prueba las predicciones de estos modelos. Fenómenos astrofísicos como los \textbf{estallidos de rayos gamma} o los \textbf{núcleos galácticos activos} podrían ser fuentes de partículas ultraenergéticas capaces de revelar estos efectos.

En resumen, los autores concluyen que los enfoques efectivos que han desarrollado permiten estudiar la \textbf{longitud mínima} con predicciones observables, y que experimentos futuros serán clave para explorar y verificar estos efectos en la naturaleza.